\documentclass[12pt]{article}
\usepackage[utf8]{inputenc}
\usepackage{geometry}
\geometry{a4paper, margin=1in}
\usepackage{hyperref}
\usepackage[english]{babel}
\usepackage{csquotes}
\usepackage[style=apa, backend=biber]{biblatex}
\addbibresource{references.bib}

\title{Literature Review}
\author{Huanrui CHEN}

\begin{document}

\maketitle

The chilling effect refers to the phenomenon of self-imposed restrictions on behavior by individuals or groups (Büchi et al., 2022). This effect can occur when people feel they are being monitored, such as being informed that their actions will be recorded and scrutinized by the government (Kappeler et al., 2023), or realizing that service providers may access their digital footprints (Strycharz \& Segijn, 2024). Research indicates that citizens who are under surveillance commonly perceive such monitoring actions as an infringement on their freedom, eliciting concerns and anxiety (Rubel, 2007). Concurrently, citizens report mitigating potential negative impacts of surveillance by reducing the use of specific software or platforms (Strycharz \& Segijn, 2024), and some respondents also report employing less controversial expressions in situations where they might be monitored, thus avoiding the risk of potentially incriminating themselves (Stoycheff et al., 2019).

However, some studies have indicated that social control and repression may provoke a backlash (Sullivan \& Davenport, 2017). Previous research suggests that individuals have certain expectations regarding their freedom to act. If these freedoms are reduced or threatened, there is an incentive for individuals to reclaim these freedoms (Brehm, 1966). Empirical research has also substantiated this theoretical framework, showing that in democratic countries, government repression and sanctions can lead to higher levels of protest and demonstration (Gupta et al., 1993). The theory of political opportunity structures provides a framework for understanding this phenomenon, noting that governments' unreasonable actions may inadvertently create new mobilization opportunities for social movements because it enhances the rationality and urgency of collective action (Meyer \& Minkoff, 2004). However, it is also important to note that in non-democratic societies, state repression may significantly reduce political opposition, as it often employs more severe and even violent means of suppression (Gupta et al., 1993).

Previous research on social movements has highlighted the potential for public resistance to societal control and repression, distinguishing between its manifestations in democratic and non-democratic countries. However, in the digital age, governments possess more means to monitor citizens, such as analyzing internet traffic to monitor and intervene in citizens' online activities (Marczak et al., 2015) or utilizing social media surveillance to gather information (Tai \& Fu, 2020). Concurrently, citizens have more flexible ways to express resistance against the government, such as posting critical content on anonymous online platforms (Tufekci, 2017) and using encrypted communication tools to evade surveillance (McCoy et al., 2008). This discrepancy suggests that governments may implement control and deterrence through more indirect and non-violent means, while citizens can retaliate using more adaptable methods. The new characteristics of the digital age imply that a fresh perspective may be required to examine the impact of government surveillance on citizen protests. Additionally, another possibility related to surveillance, the chilling effect, has been discussed by previous arguments. However, previous studies have primarily relied on surveys or experiments to collect data, leading to a potential discrepancy between people's actual behavior and their stated opinions. For instance, the privacy paradox theory highlights that despite consumers' frequent complaints about corporate invasions of their personal privacy, they are still willing to provide personal information to platforms and make their profiles public (Norberg et al., 2007). This indicates that previous research may have overly focused on public opinions rather than behaviors, and empirical studies using large-scale observations of actual online behaviors could fill the gap in understanding the impact of surveillance on citizen actions.

To investigate the impact of suppression and surveillance in the digital age on dissent, this proposal examines the changes in online discourse in Hong Kong before and after the implementation of the National Security Law, utilizing all post data from LIHKG, the largest online forum in Hong Kong, to comprehensively measure citizens' real behaviors in the digital space. The 2019 Anti-Extradition Law Amendment Bill Movement in Hong Kong is considered a paradigm of digital activism, with protestors leveraging social media to organize rallies and disseminate information (Zhong \& Zhou, 2022). Throughout this process, social media and online platforms served as tools for mobilization and communication and as arenas for an information war, presenting this decentralized social movement in a process that attracted international attention, leaving a wealth of data on platforms such as LIHKG (Ku, 2020). However, the movement concluded in 2020 following the Chinese government's enactment of the Hong Kong National Security Law on June 30, 2020. The law claims to safeguard national security within the Hong Kong Special Administrative Region by targeting acts of secession, subversion of state power, terrorist activities, and collusion with foreign forces. Under this legislation, a significant number of dissenters and political activists who participated in the movement were prosecuted by the Hong Kong government based on their speech. Research indicates that enacting the National Security Law curtailed street protests (Kobayashi et al., 2021). The unique scenario of Hong Kong's transition from a democratic to an authoritarian society, along with the control and surveillance imposed on its citizens following the law's implementation, makes Hong Kong a pertinent case study for examining the effects of digital surveillance on citizen dissent and resistance in the digital era. This discussion explores whether the National Security Law, as a legal mechanism imposing surveillance pressure on citizens, can suppress online dissent as successfully as it did street protests or whether it will further ignite public anger, prompting them to express their grievances in the anonymous spaces of the internet.

\nocite{*}
\printbibliography
\end{document}

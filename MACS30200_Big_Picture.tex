\documentclass{article}
\usepackage{tabularx} 
\usepackage{multirow} 
\usepackage{geometry} 
\usepackage{booktabs} 

\geometry{left=1cm,right=1cm,top=1cm,bottom=1cm} 

\begin{document}

\title{Big Picture: The Impact of Digital Surveillance on Online Political Discussions in Hong Kong}
\author{Huanrui CHEN}
\date{\today}
\maketitle


\noindent
\begin{tabularx}{\textwidth}{|l|X|X|}
    \hline
    \textbf{Section} & \textbf{Subsection} & \textbf{Summary} \\ \hline
    \multirow{4}{*}{Introduction} & Context & The study addresses the relationship between digital surveillance, exemplified by the Hong Kong National Security Law, and its influence on public political discussions on the internet. \\ \cline{2-3}
    & Your Research Question/Proposed Project & How does digital surveillance affect public political discussions on the Internet, particularly in the context of Hong Kong's National Security Law? \\ \cline{2-3}
    & What does the existing literature say & Previous studies have focused on public perceptions of surveillance, noting an increase in self-censorship reported by internet users under digital surveillance or a potential decrease in the frequency of expression. \\ \cline{2-3}
    & Significance with respect to existing knowledge/applied problem & There exists a gap in empirical research regarding the specific impact of surveillance measures on online political discussions. Previous studies on perceptions of digital surveillance primarily utilized surveys to understand the public's views on potential digital surveillance or employed experimental simulations to explore the potential effects of digital surveillance on netizen behavior. However, there is a lack of empirical research using large-scale observational data of actual online behavior. \\ \hline
    \multirow{2}{*}{Data and Methods} & State data/design and justify & The study will track changes in political discussion on LIHKG following the National Security Law, using web scraping and ChatGPT API for data annotation. \\ \cline{2-3}
    & State analytical method (if applicable) and justify & Time series analysis will be used to examine post frequencies and nature before and after the law's implementation. \\ \hline
    \multirow{5}{*}{Feasibility} & Evaluation of approach w.r.t. RQ/project goal & The study's approach is technically feasible with preliminary tests showing promising results using a web crawler and ChatGPT. \\ \cline{2-3}
    & Initial Results (or Mock-up) & Initial tests indicate ChatGPT's efficacy in annotating Cantonese posts accurately. \\ \cline{2-3}
    & Proposed timeline & May: Data collection; July: Data analysis; September: First draft completion. \\ \cline{2-3}
    & Securing an Advisor/Sponsor & Dr. Molly Offer-Westort, Dr. Ali Sanaei, and Dr. Nick Feamster are considered good advisor candidates. \\ \cline{2-3}
    & Cost and funding (if applicable) & ChatGPT 4.0 turbo API: \$900, ChatGPT 3.5 turbo API: \$50, Open-source models: \$0. The project is self-funded. \\ \hline
\end{tabularx}

\end{document}
